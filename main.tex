% Version 2.20 of 2017/10/04
%
\documentclass[runningheads]{llncs}
%
\usepackage{graphicx}
% If you use the hyperref package, please uncomment the following line
% to display URLs in blue roman font according to Springer's eBook style:
% \renewcommand\UrlFont{\color{blue}\rmfamily}

% \usepackage{orcidlink} % Orcid links
% Fix underscore in dois
\usepackage[strings]{underscore}

\begin{document}
%
\title{Instantaneous, understandable, and actionable soundness checking of industrial BPMN models}
%
%\titlerunning{Abbreviated paper title}
% If the paper title is too long for the running head, you can set
% an abbreviated paper title here
%
% \author{Tim Kr\"{a}uter\inst{1}\orcidlink{0000-0003-1795-0611} \and
% Adrian Rutle\inst{1}\orcidlink{0000-0002-4158-1644} \and
% Harald K\"{o}nig\inst{2,1}\orcidlink{0000-0001-6304-6311} \and
% Yngve Lamo\inst{1}\orcidlink{0000-0001-9196-1779}}
% %
% \authorrunning{T. Kräuter et al.}
% \institute{Western Norway University of Applied Sciences, Bergen, Norway 
% \email{tkra@hvl.no, aru@hvl.no, yla@hvl.no} \and
% University of Applied Sciences, FHDW, Hanover, Germany\\
% \email{harald.koenig@fhdw.de}}
%
\maketitle              % typeset the header of the contribution
%
\begin{abstract}
TODO
\keywords{
BPM \and
BPMN \and
BPMN analysis
}
\end{abstract}

% Up to 16 pages, including everything.

\section{Introduction}
% Why useful?
\cite{fahlandAnalysisDemandInstantaneous2011}
% Claims a lot of problems in workflow models, which can be found using soundness checks.
% Go over all three claims briefly: instantaneous (Definition), understandable (soundness properties are hard to understand for modelers), and actionable (comparable to quick fixes).

% Soundness stuff taken from which adopted it for BPMN
\cite{corradiniClassificationBPMNCollaborations2018}

\section{Instantaneous}
Instantaneous definition from here: \cite{fahlandAnalysisDemandInstantaneous2011}

% Provide empirical evidence that it is instantaneous. Use repositories and synthetic benchmarks with increasing model size and degree of parallelity. 
% Add all those datasets with artifacts using a Zenodo link --> Artifacts badge

These guys have some benchmarks:
\cite{corradiniFormalApproachAnalysis2021}

\section{Understandable}

% Highlight directly and instantly in the diagram --> Overlays (and colors)
% Interactive visualization of counterexamples.
\cite{camundaservicesgmbhBpmnjsTokenSimulation2023}

\section{Actionable}

% Quick fix providers/analysis resolvers coded for the different soundness properties.
% Custom quick fix providers are also possible in the future with access to model-checking

\section{Implementation}

\subsection{Rust}
% Why in Rust?
% Low-level language with modern features.
% Highlight Rusts speed and safety claims
% Good for CLI tools such as this
% Zero-copy?
% Zero-overhead abstractions --> related to speed
% Reimplement in rust meme

\subsection{BPMN semantics}
% Describe the runtime model. What is a state? --> Draw a UML diagram
% Describe what the initial state is. --> Could be configurable in the future.
% Describe BPMN semantics with some diagrams for some examples.
% Use a good running example with a modeling error that wasn't used before (double-blind).
\subsection{Description}
% Tool UI with screenshot
\cite{camundaservicesgmbhBpmnjs2023}
% Tool architecture? UI in JS and backend in Rust.
% Just the model checker as a CLI app or a CLI that starts the whole thing as a service.

\section{Related work}

% Different ways of comparison
% 1. BPMN features supported (sub of 2.)
% 2. Capabilities (Which soundness properties are supported, custom properties, counter-example visualization, tool maturity, tool integration, actionability of the tool, etc.)
% 3. Performance using different benchmarks. --> Needs its own paper where everyone can contribute.

\cite{krauterFormalizationAnalysisBPMN2023} % Also add LMCS citation here.

\cite{vangorpVisualTokenbasedFormalization2013}

\cite{corradiniBProVeToolSupport2017,corradiniFormalApproachAnalysis2021}

\section{Limitations \& Threats to Validity}

\section{Conclusion \& Future work}

\bibliographystyle{splncs04} 
\bibliography{bib}

\end{document}
