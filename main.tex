% This is samplepaper.tex, a sample chapter demonstrating the
% LLNCS macro package for Springer Computer Science proceedings;
% Version 2.20 of 2017/10/04
%
\documentclass[runningheads]{llncs}
%
\usepackage{graphicx}
% Used for displaying a sample figure. If possible, figure files should
% be included in EPS format.
%
% If you use the hyperref package, please uncomment the following line
% to display URLs in blue roman font according to Springer's eBook style:
% \renewcommand\UrlFont{\color{blue}\rmfamily}

% \usepackage{orcidlink} % Orcid links
% Fix underscore in dois
\usepackage[strings]{underscore}

\begin{document}
%
\title{Instantaneous, understandable, and actionable soundness checking of industrial BPMN models}
%
%\titlerunning{Abbreviated paper title}
% If the paper title is too long for the running head, you can set
% an abbreviated paper title here
%
% \author{Tim Kr\"{a}uter\inst{1}\orcidlink{0000-0003-1795-0611} \and
% Adrian Rutle\inst{1}\orcidlink{0000-0002-4158-1644} \and
% Harald K\"{o}nig\inst{2,1}\orcidlink{0000-0001-6304-6311} \and
% Yngve Lamo\inst{1}\orcidlink{0000-0001-9196-1779}}
% %
% \authorrunning{T. Kräuter et al.}
% \institute{Western Norway University of Applied Sciences, Bergen, Norway 
% \email{tkra@hvl.no, aru@hvl.no, yla@hvl.no} \and
% University of Applied Sciences, FHDW, Hanover, Germany\\
% \email{harald.koenig@fhdw.de}}
%
\maketitle              % typeset the header of the contribution
%
\begin{abstract}
TODO
\keywords{
BPM \and
BPMN \and
BPMN analysis
}
\end{abstract}

\section{Introduction}

\cite{corradiniClassificationBPMNCollaborations2018}

\section{Instantaneous}
Instantaneous definition from here: \cite{fahlandAnalysisDemandInstantaneous2011}

% Provide empirical evidence that it is instantaneous. Use repositories and synthetic benchmarks with increasing model size and degree of parallelity. 

These guys have some benchmarks:
\cite{corradiniFormalApproachAnalysis2021}

\section{Understandable}
\section{Actionable}

\section{Tool implementation}
\subsection{Rust}

\section{Related work}

% Different ways of comparison
% 1. BPMN features supported (sub of 2.)
% 2. Capabilities (Which soundness properties are supported, custom properties, counter-example visualization, tool maturity, tool integration, actionability of the tool, etc.)
% 3. Performance using different benchmarks. --> Needs its own paper where everyone can contribute.

\cite{krauterFormalizationAnalysisBPMN2023}

\cite{vangorpVisualTokenbasedFormalization2013}

\cite{corradiniBProVeToolSupport2017,corradiniFormalApproachAnalysis2021}

\section{Limitations \& Threats to Validity}

\section{Conclusion \& Future work}

\bibliographystyle{splncs04} 
\bibliography{bib}

\end{document}
